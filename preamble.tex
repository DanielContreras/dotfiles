% Some basic packages
\usepackage[margin=1in]{geometry}
\usepackage[utf8]{inputenc}
\usepackage[T1]{fontenc}
\usepackage{textcomp}
\usepackage[english]{babel}
\usepackage{url}
\usepackage{graphicx,wrapfig}
\usepackage{float}
\usepackage{booktabs}
\usepackage{enumitem}
\usepackage{tikz-cd}
\usepackage{hyperref, bookmark, imakeidx}
\usepackage{cite}
\usepackage{pgfplots}
\usepackage{listings}
\usepackage{color}

%%%%%%%%%%%%%%% source code %%%%%%%%%%%%%%%%%%%%%%%%
\definecolor{dkgreen}{rgb}{0,0.6,0}
\definecolor{gray}{rgb}{0.5,0.5,0.5}
\definecolor{mauve}{rgb}{0.58,0,0.82}

\lstset{frame=tb,
  language=R,
  aboveskip=3mm,
  belowskip=3mm,
  showstringspaces=false,
  columns=flexible,
  basicstyle={\small\ttfamily},
  numbers=none,
  numberstyle=\tiny\color{gray},
  keywordstyle=\color{blue},
  commentstyle=\color{dkgreen},
  stringstyle=\color{mauve},
  breaklines=true,
  breakatwhitespace=true,
  tabsize=3
}
%%%%%%%%%%%%%%%%%%%%%%%%%%%%%%%%%%%%%%%%%%%%%%%%%%%%%

\graphicspath{{./figures/}}
%\tolerance=1
%\emergencystretch=\maxdimen
%\hyphenpenalty=10000
%\hbadness=10000

% Don't indent paragraphs, leave some space between them
\usepackage{parskip}

% Hide page number when page is empty
\usepackage{emptypage}
\usepackage{subcaption}
\usepackage{multicol}
\usepackage{xcolor}

% Math stuff
\usepackage{amsmath, amsfonts, mathtools, amsthm, amssymb, scrextend, commath, ifthen}
\usepackage{coolstr, coollist, forloop, cool}
% Fancy script capitals
\usepackage{mathrsfs}
\usepackage{cancel}
% Bold math
\usepackage{bm}
% Some shortcuts
\newcommand\N{\ensuremath{\mathbb{N}}}
\newcommand\R{\ensuremath{\mathbb{R}}}
\newcommand\Z{\ensuremath{\mathbb{Z}}}
\renewcommand\O{\ensuremath{\emptyset}}
\newcommand\Q{\ensuremath{\mathbb{Q}}}
\newcommand\C{\ensuremath{\mathbb{C}}}
\newcommand\K{\ensuremath{\mathbb{K}}}
\newcommand\M{\ensuremath{\mathbb{M}}}
\newcommand\p{\ensuremath{\mathbb{P}}}
\newcommand{\eq}[2]{\mathrel{\operatorname*{=}_#1^#2}}
\newcommand\inv[1]{#1\raisebox{1.15ex}{$\scriptscriptstyle-\!1$}}
\newcommand\divides{\ensuremath{\bigm|}}
\newcommand*{\everymodeprime}{\ensuremath{\prime}}

% Put x \to \infty below \lim
\let\svlim\lim\def\lim{\svlim\limits}

%Make implies and impliedby shorter
\let\implies\Rightarrow
\let\impliedby\Leftarrow
\let\iff\Leftrightarrow
\let\epsilon\varepsilon

% Add \contra symbol to denote contradiction
\usepackage{stmaryrd} % for \lightning
\newcommand\contra{\scalebox{1.5}{$\lightning$}}

% Make left-brace and right-brace shorter
\newcommand{\lbr}{\lbrace}
\newcommand{\rbr}{\rbrace}

%% From Abstract Algebra 
\RequirePackage{bbm}
\RequirePackage{lipsum}
\RequirePackage{extarrows}
\RequirePackage{makeidx}
\RequirePackage{graphicx}
\RequirePackage{subfiles}
\RequirePackage{soul}
\RequirePackage{comment}

\newcommand{\ind}[1]{\index{#1}\textbf{#1}}
\newcommand{\exercise}{Left as an exercise to the reader.}
\newcommand{\pexercise}{\begin{proof}\exercise\end{proof}}

%% brakcet macros
\newcommand{\abra}[1]{\ensuremath{\left\langle#1\right\rangle}}
\newcommand{\cbra}[1]{\left\{#1\right\}}
\newcommand{\pbra}[1]{\left(#1\right)}
\newcommand{\sbra}[1]{\ensuremath{\left[#1\right]}}
\newcommand{\vbra}[1]{\ensuremath{\left\lvert #1 \right\rvert}}

%% bb shortcuts
\newcommand{\bbz}{\ensuremath{\mathbb{Z}}}
\newcommand{\bbr}{\ensuremath{\mathbb{R}}}
\newcommand{\bbc}{\ensuremath{\mathbb{C}}}
\newcommand{\bbq}{\ensuremath{\mathbb{Q}}}
\newcommand{\bbn}{\ensuremath{\mathbb{N}}}
\newcommand{\bbf}{\ensuremath{\mathbb{F}}}

%% style macros
\newcommand{\sfrac}[2]{\,^{#1}\!/_{#2}}
\newcommand{\barrow}{\ensuremath{\xrightarrow{\sim}}}
\renewcommand{\underline}[1]{\ul{#1}}

%% references
\newcommand{\tref}[1]{Theorem \ref{#1}}
\newcommand{\cref}[1]{Corollary \ref{#1}}
\newcommand{\lref}[1]{Lemma \ref{#1}}
\newcommand{\aref}[1]{Application \ref{#1}}
\newcommand{\pref}[1]{Proposition \ref{#1}}
\newcommand{\clref}[1]{Claim \ref{#1}}
\newcommand{\fref}[1]{Figure \ref{fig:#1}}
\newcommand{\eref}[1]{Example \ref{#1}}
\newcommand{\dref}[1]{Definition \ref{#1}}
\newcommand{\prref}[2]{Problem \ref{pr:#1}.\ref{pr:#1.#2}}

% Complex Analysis Operators
\newcommand{\im}[1]{\ensuremath{\operatorname{Im}(#1)}}
\newcommand{\re}[1]{\ensuremath{\operatorname{Re}(#1)}}
\renewcommand{\Arg}[1]{\ensuremath{\operatorname{Arg}#1}}
\newcommand{\plog}[1]{\ensuremath{\operatorname{Log}#1}}

% Statistics Operators
\newcommand{\Ex}[1]{\ensuremath{\operatorname{E}\sbr{#1}}}
\newcommand{\Var}[1]{\ensuremath{\operatorname{Var}\sbr{#1}}}
\newcommand{\Std}[1]{\ensuremath{\operatorname{Std}\sbr{#1}}}
\newcommand{\Skew}[1]{\ensuremath{\operatorname{Skew}\sbr{#1}}}
\newcommand{\Kurt}[1]{\ensuremath{\operatorname{Kurt}\sbr{#1}}}
\newcommand{\Cor}[2]{\ensuremath{\operatorname{Corr}\del{#1,#2}}}
\newcommand{\Cov}[2]{\ensuremath{\operatorname{Cov}\del{#1,#2}}}
\newcommand{\Bias}[2]{\ensuremath{\operatorname{Bias}\del{#1,#2}}}
\newcommand{\GammaDist}[3]{\ensuremath{#1\sim\operatorname{Gamma}\del{#2,#3}}}
\newcommand{\NormDist}[3]{\ensuremath{#1\sim\operatorname{Normal}\del{#2,#3}}}
\newcommand{\BinDist}[3]{\ensuremath{#1\sim\operatorname{Binomial}\del{#2,#3}}}
\newcommand{\BetaDist}[3]{\ensuremath{#1\sim\operatorname{Beta}\del{#2,#3}}}
\newcommand{\UniDist}[3]{\ensuremath{#1\sim\operatorname{Uniform}\del{#2,#3}}}
\newcommand{\ChiDist}[2]{\ensuremath{#1\sim\operatorname{\chi}^2\del{#2}}}
\newcommand{\TDist}[2]{\ensuremath{#1\sim\operatorname{T}\del{#2}}}
\newcommand{\FDist}[3]{\ensuremath{#1\sim\operatorname{F}\del{#2,#3}}}
\newcommand{\BerDist}[2]{\ensuremath{#1\sim\operatorname{Bernoulli}\del{#2}}}
\newcommand{\GeoDist}[2]{\ensuremath{#1\sim\operatorname{Geometric}\del{#2}}}
\newcommand{\PoiDist}[2]{\ensuremath{#1\sim\operatorname{Poisson}\del{#2}}}
\newcommand{\ExpDist}[2]{\ensuremath{#1\sim\operatorname{Exponential}\del{#2}}}
\newcommand{\T}[1]{\ensuremath{\operatorname{T}_{#1}}}
\newcommand{\F}[2]{\ensuremath{\operatorname{F}_{#1,#2}}}
\newcommand{\X}[1]{\ensuremath{\operatorname{\chi}^2_{#1}}}

% Algebra Operators
\newcommand{\sign}[1]{\ensuremath{\operatorname{sgn}#1}}
\newcommand{\image}[1]{\ensuremath{\operatorname{im}#1}}
\newcommand{\aut}[1]{\ensuremath{\operatorname{Aut}(#1)}}
\newcommand{\inn}[1]{\ensuremath{\operatorname{Inn}(#1)}}
\newcommand{\End}[1]{\ensuremath{\operatorname{End}(#1)}}
\newcommand{\tor}[1]{\ensuremath{\operatorname{Tor}(#1)}}
\newcommand{\ann}[1]{\ensuremath{\operatorname{Ann}(#1)}}
\newcommand{\gal}[1]{\ensuremath{\operatorname{Gal}()#1)}}
\newcommand{\supp}[1]{\ensuremath{\operatorname{Supp}(#1)}}
\newcommand{\stab}[1]{\ensuremath{\operatorname{Stab}(#1)}}

% Linear Algebra Operators
\newcommand{\tr}[1]{\ensuremath{\operatorname{tr}(#1)}}
\newcommand{\rank}[1]{\ensuremath{\operatorname{rank}#1}}
\newcommand{\Span}[1]{\ensuremath{\operatorname{span}#1}}
\newcommand{\row}[1]{\ensuremath{\operatorname{row}#1}}
\newcommand{\col}[1]{\ensuremath{\operatorname{col}#1}}
\newcommand{\nul}[1]{\ensuremath{\operatorname{null}#1}}
\newcommand{\nullity}[1]{\ensuremath{\operatorname{nullity}#1}}
\newcommand{\proj}[1]{\ensuremath{\operatorname{proj}#1}}
\newcommand{\dist}[1]{\ensuremath{\operatorname{d}(#1)}}

% Calculus Operators
\newcommand{\dom}[1]{\ensuremath{\operatorname{dom}#1}}
\newcommand{\lub}[1]{\ensuremath{\operatorname{lub}\del{#1}}}
\newcommand{\bvec}[1]{\mathbf{#1}}
\newcommand{\bhat}[1]{\mathbf{\hat{#1}}}
\DeclareMathOperator{\curve}{\mathscr{C}}
\DeclareMathOperator{\Lagr}{\mathcal{L}}
\DeclareMathOperator{\Lapl}{\mathcal{L}}
\DeclareMathOperator{\hess}{\bm{\mathscr{H}}}

%% algebra macros
\newcommand{\map}[3]{\ensuremath{{#1 : #2 \rightarrow #3}}}
\newcommand{\xmap}[3]{\ensuremath{{#2 \xrightarrow{#1} #3}}}
\newcommand{\qgrp}[2]{{\sfrac{#1}{#2}}}
\newcommand{\GZmod}[1][n]{\sfrac{\bbz}{#1\bbz}}
\newcommand{\pGZmod}[1][n]{\pbra{\GZmod[#1]}}
\DeclareMathOperator{\Char}{Char}
\DeclareMathOperator{\ev}{ev}
\DeclarePairedDelimiter\card{\lvert}{\rvert}%
\DeclareMathOperator{\Frac}{Frac}
\DeclareMathOperator{\degree}{degree}
\newcommand{\poly}[2]{\ensuremath{\mathcal{P}\pbra{#1, #2}}}
\newcommand{\id}[1][]{\ensuremath{\iota_{#1}}}
\newcommand{\ida}[1][]{\ensuremath{\id[#1]^{+}}}
\newcommand{\idm}[1][]{\ensuremath{\id[#1]^{*}}}
\newcommand{\divs}[2]{\ensuremath{#1 \divides #2 }}
\newcommand{\order}[1]{\ensuremath{o(#1)}}
\newcommand{\Oind}[2]{\ensuremath{\vbra{ #1:#2}}}
\newcommand{\psign}[1]{\ensuremath{\textrm{sign}\pbra{#1}}}
\newcommand{\ngrp}{\ensuremath{\trianglelefteq}}
\newcommand{\ngrpeq}{\ensuremath{\trianglelefteq}}
\newcommand{\cset}[2]{\ensuremath{\cbra{#1\ \middle |\ #2}}}
\renewcommand{\st}{\ensuremath{\text{ such that }}}
\newcommand{\gen}[1]{\ensuremath{\langle #1 \rangle}}
\newcommand{\normal}[2]{#1\trianglelefteq #2}
\newcommand{\seq}[2]{\ensuremath{#1_1, #1_2, \ldots , #1_{#2}}}
\newcommand{\sequence}[3]{\ensuremath{#1_1 #2 #1_2 #2 \ldots #2 #1_{#3}}}
\newcommand{\infseq}[2]{\ensuremath{#1_1 #2 #1_2 #2 \ldots}}
\newcommand{\sseq}[4]{\ensuremath{#1_1,\, #1_2,\, \ldots ,\, #1_{#2}}}
\newcommand{\mseq}[3]{\ensuremath{#1_{1}#2_{1},\, #1_{2}#2_{2},\, \ldots ,\, #1_{#3}#2_{#3}}}
\newcommand{\amser}[3]{\ensuremath{#1_{1}#2_{1}+#1_{2}#2_{2}+\ldots +#1_{#3}#2_{#3}}}
\newcommand{\samser}[3]{\ensuremath{#1_0+#1_{1}#2+\ldots +#1_{#3}#2^{#3}}}

\newcommand{\GL}[2][n]{\ensuremath{\mathcal{GL}_{#1}}\pbra{#2}}
\newcommand{\SL}[2][n]{\ensuremath{\mathcal{SL}_{#1}}\pbra{#2}}
\newcommand{\mset}[1][m \times n]{\ensuremath{\mathcal{M}_{#1}}}
\newcommand{\sqmset}[1][n]{\ensuremath{\mset[#1 \times #1]}}

\newcommand{\dfwd}{$\Rightarrow$\ }
\newcommand{\drev}{$\Leftarrow$\ }

\newcommand{\lcsr}[2]{\ensuremath{{}_{#1}{#2}}}
\newcommand{\rcsr}[2]{\ensuremath{{#2}_{#1}}}

\newcommand{\oddP}[1][n]{\ensuremath{\textrm{Odd}_{#1}}}
\newcommand{\evenP}[1][n]{\ensuremath{\textrm{Even}_{#1}}}

\newcommand{\bmat}[1]{\ensuremath{\begin{bmatrix}#1\end{bmatrix}}}
\newcommand{\vmat}[1]{\ensuremath{\begin{vmatrix}#1\end{vmatrix}}}
\newcommand{\pmat}[1]{\ensuremath{\begin{pmatrix}#1\end{pmatrix}}}

% Extends matrix environments to give it functionality to become an augmented 
% matrix. Usage:
% \begin{bmatrix}[cc|c]
%   a & c & d \\
%   d & e & f
% \end{bmatrix}
\makeatletter
\renewcommand*\env@matrix[1][*\c@MaxMatrixCols c]{%
  \hskip -\arraycolsep
  \let\@ifnextchar\new@ifnextchar
  \array{#1}}
\makeatother

\newcommand{\renewcounters}{
    \setcounter{definition}{0}\setcounter{lemma}{0}\setcounter{theorem}{0}\setcounter{corollary}{0}\setcounter{prop}{0}
}

% Log style
\Style{LogParen=p}
\Style{DDisplayFunc=inset,DShorten=true}

% Evaluated from a to b
\newcommand*\Eval[3]{\left.#1\right\rvert_{#2}^{#3}}

% horizontal rule
\newcommand\hr{
    \noindent\rule[0.5ex]{\linewidth}{0.5pt}
}

% hide parts
\newcommand\hide[1]{}

% si unitx
\usepackage{siunitx}
\sisetup{locale = FR}

% Environments
\makeatother
% For box around Definition, Theorem, \ldots
\usepackage{mdframed}
\mdfsetup{skipabove=1em,skipbelow=0em}
\theoremstyle{definition}
\newtheorem*{eg}{Example}
\newtheorem{counterexample}{Counterexample}[section]
\newtheorem*{notation}{Notation}
\newtheorem*{previouslyseen}{As previously seen}
\newtheorem*{remark}{Remark}
\newtheorem*{note}{Note}
\newtheorem*{claim}{Claim}
\newtheorem{application}{Application}[section]
\newtheorem*{problem}{Problem}
\newtheorem*{solution}{Solution}
\newtheorem*{rul}{Rule}
\newtheorem*{observe}{Observe}
\newtheorem*{property}{Properties}
\newtheorem*{intuition}{Intuition}
\newmdtheoremenv[nobreak=true]{definition}{Definition}
\newmdtheoremenv[nobreak=true]{prop}{Proposition}
\newmdtheoremenv[nobreak=true]{properties}{Properties}
\newmdtheoremenv[nobreak=true]{theorem}{Theorem}
\newmdtheoremenv[nobreak=true]{corollary}{Corollary}
\newmdtheoremenv[nobreak=true]{consequence}{Consequence}
\newmdtheoremenv[nobreak=true]{lemma}{Lemma}

\newtheorem*{examplex}{Example}
\newenvironment{example}
  {\pushQED{\qed}\renewcommand{\qedsymbol}{$\triangle$}\examplex}
  {\popQED\endexamplex}

% End example and intermezzo environments with a small diamond (just like proof
% environments end with a small square)
\usepackage{etoolbox}
\AtEndEnvironment{vb}{\null\hfill$\diamond$}%
\AtEndEnvironment{intermezzo}{\null\hfill$\diamond$}%
% \AtEndEnvironment{opmerking}{\null\hfill$\diamond$}%


% % Fix some spacing
% % http://tex.stackexchange.com/questions/22119/how-can-i-change-the-spacing-before-theorems-with-amsthm
\makeatletter
\def\thm@space@setup{%
  \thm@preskip=\parskip \thm@postskip=0pt
}

\usepackage{xifthen}
\def\testdateparts#1{\dateparts#1\relax}
\def\dateparts#1 #2 #3 #4 #5\relax{
    \marginpar{\small\textsf{\mbox{#1 #2 #3 #5}}}
}

%fakesection
\newcommand{\fakesection}[1]{%
  \par\refstepcounter{section}% Increase section counter
  \sectionmark{#1}% Add section mark (header)
  \addcontentsline{toc}{section}{\protect\numberline{\thesection}#1}% Add section to ToC
  % Add more content here, if needed.
}
%fakesubsection
\newcommand{\fakesubsection}[1]{%
  \par\refstepcounter{subsection}% Increase subsection counter
  \subsectionmark{#1}% Add subsection mark (header)
  \addcontentsline{toc}{subsection}{\protect\numberline{\thesubsection}#1}% Add subsection to ToC
  % Add more content here, if needed.
}

%lecture macro
\def\@lecture{}%
\newcommand{\lecture}[3]{
    \ifthenelse{\isempty{#3}}{%
        \def\@lecture{Lecture #1}%
    }{%
        \def\@lecture{Lecture #1: #3}%
    }%
    \subsection*{\@lecture}
    \marginpar{\small\textsf{\mbox{#2}}}
}

% % These are the fancy headers
\usepackage{fancyhdr}
\pagestyle{fancy}

% LE: left even
% RO: right odd
% CE, CO: center even, center odd

\fancyhead[RO,LE]{\@lecture} % Right odd,  Left even
\fancyhead[RE,LO]{}          % Right even, Left odd

\fancyfoot[RO,LE]{\thepage}  % Right odd,  Left even
\fancyfoot[RE,LO]{}          % Right even, Left odd
\fancyfoot[C]{\leftmark}     % Center

\makeatother

% Todonotes and inline notes in fancy boxes
\usepackage{todonotes}
\usepackage{tcolorbox}

% Make boxes breakable
\tcbuselibrary{breakable}

% Figure support as explained in my blog post.
\usepackage{import}
\usepackage{pdfpages}
\usepackage{transparent}

\newcommand{\incfig}[2][1]{%
    \def\svgwidth{#1\columnwidth}
    \import{./figures/}{#2.pdf_tex}
}

% Fix some stuff
% %http://tex.stackexchange.com/questions/76273/multiple-pdfs-with-page-group-included-in-a-single-page-warning
\pdfsuppresswarningpagegroup=1

\makeindex[intoc]

% Hyperlink and bookmark setup
%\hypersetup{
%    colorlinks=true,
%    linkcolor=blue,
%    filecolor=magenta,
%    urlcolor=cyan,
%    pdftitle={Complex Analysis Lecture Notes},
%    bookmarks=true,
%    pdfstartview=
%}
%\bookmarksetup{
%  numbered,
%  open
%}

% My name
\author{Daniel Contreras \\ 10080311}
